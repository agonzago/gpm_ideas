%% LyX 2.4.0~RC3 created this file.  For more info, see https://www.lyx.org/.
%% Do not edit unless you really know what you are doing.
\documentclass[american]{article}
\usepackage[T1]{fontenc}
\usepackage[utf8]{inputenc}
\usepackage{amsmath}
\usepackage{babel}
\begin{document}
\begin{enumerate}
\item I'm working on a code that reads a dynare file (of a linear model).
It can be any text file (so dynare is not fundamental). The idea is
that, I read the dyanre-like file and create a json file with the
equations written in timing t and t+1 as will be expected by Klein's
method. The model has long leads and lags so I need to create auxiliary
equations and variables to accommodate this feature. After creating
the json file I compute the Jacobbian the equation with respect to
the variables in t and t+1 using analytical methods and save the jacobbian's
in the file that letter is loaded in memory. The procedure loads these
files and evaluate them a parameter value and call Klein's solution
method. The solution is working fine. 
\item The system of equations I'm working on is

\[
AE_{t}x_{t+1}=Bx_{t}+C\epsilon_{t}
\]
where $x_{t}$ is a vector organized such that you have state first
and control after. Let's call controls $c_{t}$ and states $s_{t}$.
States included lagged values of the endogenous variables and lagged
values of the exogenous variables. 
\item As mentioned before any variable in the model could have long lead
and lags. For these variables the parser generates auxiliary variables
and equations to write the model in the standard Klein format. So
if there is a lead variable with orrder 2 for example the parser will
create the following variables and equations. Say there is a variable
in the system with y(+2) that is with two leads in this case, 

\begin{align*}
x_{t}^{lead1} & =y(+1)\\
x_{t}^{lead2} & =x_{t+1}^{lead1}
\end{align*}
then effectively $x_{t}^{lead2}=y(+1)$, longer leads will follow
the same approach. If you longer lags, say y(-2), you will also create
auxiliary variables, in this case

\[
x_{t+1}^{lag}=y
\]
so $x_{t}^{lag}=y(-1),$similarly longer lags can be accommodated. 
\item One aspect that deserves attention is the following: In the model,
not all exogenous variables have the same lag structure. For example,
you could you have two exogenous variables 
\[
z_{1t}=\rho_{10}z_{1t-1}+\rho_{11}z_{1t-2}+\epsilon_{t}^{z1}
\]

\[
z_{2t}=\rho_{20}z_{2t-1}+\epsilon_{t}^{z2}
\]
so in this case, the $z_{1t}$ variable will be represented in Klein's
Canonical representation with the equations
\begin{align*}
x_{t+1}^{lag} & =z_{1t}
\end{align*}
so the representation would be 
\begin{align*}
z_{1t+1} & =\rho_{10}z_{t}+\rho_{11}x_{t}^{lag}+\epsilon_{t+1}^{z1}\\
x_{t+1}^{lag} & =z_{1t}
\end{align*}
that leads to the following state vector $\left(z_{1t},x_{t}^{lag},x_{t}^{lag2}\right)^{\prime}$
and associated Klein's Canonical representation 
\[
\left[\begin{array}{ccc}
1 & 0 & 0\\
0 & 1 & 0\\
0 & 0 & 1
\end{array}\right]\left[\begin{array}{c}
z_{1t+1}\\
x_{t+1}^{lag}\\
z_{2t+1}
\end{array}\right]=\left[\begin{array}{ccc}
\rho_{10} & \rho_{11} & 0\\
1 & 0 & 0\\
0 & 0 & \rho_{20}
\end{array}\right]\left[\begin{array}{c}
z_{1t}\\
x_{t}^{lag}\\
z_{2t}
\end{array}\right]+\left[\begin{array}{cc}
1 & 0\\
0 & 0\\
0 & 1
\end{array}\right]\left[\begin{array}{c}
\epsilon_{t+1}^{z1}\\
\epsilon_{t+1}^{z2}
\end{array}\right]
\]
that is, two important things emerge: 
\begin{enumerate}
\item Even if the model in dynare syntax is written as $z_{1t}=\rho_{10}z_{1t-1}+\rho_{11}z_{1t-1}+\epsilon_{t}^{z1}$
the parser should understand it as $z_{1t+1}=\rho_{10}z_{1t}+\rho_{11}z_{1t-1}+\epsilon_{t+1}^{z1}$
and $z_{t}$ is still the contemporaneous exogenous variable. So that
at shock at time $z_{t}$ moves $z_{t}$ and all other variables in
the system. Say 
\[
\tilde{y}_{t}=\alpha\tilde{y}_{t+1}+z_{t}
\]
where $\tilde{y}_{t}$ is an endogenous control. 
\item In the system above (with the Canonical representation for the shocks)
there is matrix R multiplying the vector of exogenous shocks in the
example 
\[
R=\left[\begin{array}{cc}
1 & 0\\
0 & 0\\
0 & 1
\end{array}\right]
\]
this matrix is part of matrix $C$ in the Canonical representation
and should map shocks to exogenous states. Not all states will get
a shock by construction.
\end{enumerate}
\item The next task is to build a state space representation for the Kalman
filter. The way I'm doing this is 

$s_{t+1}=Ps_{t}$

$c_{t}=Fs_{t}$

For simplicity it is easier to split the matrices of the system taking
into account the kind of variable we have (endogenous states, exogenous
state, and controls). 
\item When computing the jacobbian of the model equations variables are
ordered as follows 
\begin{verbatim}
variables = self.state_variables + self.control_variables
\end{verbatim}
where 
\begin{verbatim}
self.state_variables = endogenous_states + exo_with_shocks + exo_without_shocks
\end{verbatim}
So we can use this order to group variables in the F and P matrices
cumming from Klein's solution

Lets call $c_{t}$ controls, $k_{t}$ endogenous states and $\xi_{t}$
exogenous states. That $c_{t}$ is the vector containing self.control\_variables,
$k_{t}$ a vector containing endogenous\_states and $\xi_{t}$ ordered
as exo\_with\_shocks + exo\_without\_shocks. With this notation is
possible to expressed Klein's solution as 

\[
\left(\begin{array}{c}
k_{t+1}\\
\xi_{t+1}
\end{array}\right)=\left(\begin{array}{cc}
P_{kk} & P_{k\xi}\\
0 & P_{\xi\xi}
\end{array}\right)\left(\begin{array}{c}
k_{t}\\
\xi_{t}
\end{array}\right)+H\epsilon_{t+1}
\]
where $\epsilon_{t}$ contains the exogenous shocks and $H$ holds
1 and zeros selecting and it's function of $R$. I'm using $\xi_{t}$
to denote the exogenous states to make the point that given the lag
structure of the exiguous process $\xi$ is not equal to $z_{t}$
but it could include auxiliary states. 

\[
c_{t}=\left(\begin{array}{cc}
F_{ck} & F_{c\xi}\end{array}\right)\left(\begin{array}{c}
k_{t}\\
\xi_{t}
\end{array}\right)
\]

\item How to get the steady space representation 
\begin{align*}
k_{t+1} & =p_{kk}k_{t}+p_{k\xi}\xi_{t}\\
c_{t} & =F_{ck}k_{t}+F_{\xi\xi}\xi_{t}
\end{align*}
if I lag $\xi_{t}$ then,

\begin{align*}
k_{t+1} & =p_{kk}k_{t}+p_{k\xi}p_{\xi\xi}\xi_{t-1}+p_{k\xi}R\epsilon_{t}\\
c_{t}= & F_{ck}k_{t}+F_{\xi\xi}p_{\xi\xi}\xi_{t-1}+F_{\xi\xi}R\epsilon_{t}\\
\xi_{t}= & p_{\xi\xi}\xi_{t-1}+R\epsilon_{t}
\end{align*}
in this representation a shock today can affect, the endogenous states
tomorrow, controls and exogenous states contemporaneously.
\item I'm working on an implementation that augments the state space by
including stochastic trends and also has other characteristics. 
\begin{enumerate}
\item Only a subset of the variables is observable, as defined by the user
\item The code is written in a way the maximizes efficiency and limits the
regeneration of matrices that are constant across parameter changes.
That is, matrices, or submatrices that remain constant when parameters
are changed. This is to generate a code that is efficient when computing
the likelihood function or carry out mcmc. 
\item The model state space representation remains similar but includes
trends so the number of states is larger by definition. The augmented
state space model should look like
\begin{align*}
y_{t}^{obs} & =HC_{aug}x_{t}^{aug}\\
x_{t}^{aug} & =A_{aug}x_{t-1}^{aug}+B_{aug}\epsilon_{t}^{aug}
\end{align*}
where H: is a selection matrix with zeros and ones (or may be replace
by an index) that selects the rows of $C_{aug}x_{t}^{aug}$ that are
observable, 
\[
A_{aug}=\left[\begin{array}{cc}
A & 0\\
0 & A_{trend}
\end{array}\right]
\]

\[
B_{aug}=\left[\begin{array}{cc}
B & 0\\
0 & B_{trend}
\end{array}\right]
\]
and $\epsilon_{t}^{aug}=\left(\epsilon_{t},\epsilon_{t}^{trends}\right)$. 
\item Not all observable variables have the same trend specification. The
trend specification of a variables can be 
\begin{enumerate}
\item Random walk: 

\[
trend_{t}=trend_{t-1}+\epsilon_{t}^{level}
\]

\item Second difference
\begin{align*}
trend_{t} & =trend_{t-1}+g_{t-1}+\epsilon_{t}^{level}\\
g_{t} & =g_{t-1}+\epsilon_{t}^{growth}
\end{align*}
\item Constant mean

\[
trend_{t}=trend_{t-1}
\]

\end{enumerate}
\end{enumerate}
\item The code is structure as follows:
\begin{enumerate}
\item DynareParser just reads the file and produce some auxiliary function
and constants needed for the solution
\item ModelSover. Loads files created by the parser and computes the klain
solution 
\item AugmentedState Space creates the augmented representation. 
\end{enumerate}
\item Tasks
\begin{enumerate}
\item Check the parser and make sure it correctly creates the auxiliary
equations as explained before. 
\item Given this jaccobians check that the state space representation is
correct
\item Check the augmented class and link it to the model solver. 
\item Create a function that computes the IRFs of the simple model and the
augmented model (they should produce the same results for shocks in
the econonomic model) 
\end{enumerate}
\end{enumerate}

\end{document}
